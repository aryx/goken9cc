\documentclass{article}


\title{
Goken9cc: The Plan 9 toolchain Reborn
}
\author{
Yoann Padioleau\\
yoann.padioleau@gmail.com
}

\begin{document}
\maketitle

\begin{abstract}
An abstract
\end{abstract}

\section{Introduction}

% Plan9 didn't just rethink the kernel, graphics + network
% and windowing system, also rethink tools such as toolchain!
% assembler/linker/compilers written from scratch, by none other
% than ken thompson! Original design with assembler doing not much
% and machine code actually in linker, which allows to factorize/simplify
% code. 
% I call the set of assemblers/linkers/compilers for different archs
% kencc. In plan9 it had 386, arm, mips, ...
% Quite small! compared to gcc/binutils, clang/llvm.
% And super easy to cross compile from one arch to the other
% (now popularized by ziglang).

% Life after plan9, it evolved (see next section). Was actually used
% to bootstrap golang! The 6/8/5 assemblers/linkers/compilers were
% in go repo until 1.3 and gradually got converted to Go.
% When in Go kencc got good improvements when used
% inside golang (Elf Linux, Mach-o macos, PE windows, dwarf gdb)!
% In fact hidden inside golang is this gem
% with multi-arch AND multi-os!! can cross compile on 386 Linux
% a macos amd64 binary! or windows! with minilibc! statically!
% (again now popularized with ziglang, golang).
% Unfortunately disappeared. Hidden gem!
% And got forked and mostly lost.
% Sad because fantastic toolchain; actually I got trouble
% cross compiling xv6 on my machine with gcc/binutils!

% Time to offer a real alternative to gcc/binutils and clang/llvm/lld!
% Lots of efforts on linker (mold, ...) but again time to go back
% to simplicity! and re-expose this hidden plan9 and inferno and golang gem!!
 
%TOC:
% - kencc history?
% - kencc fork history
%   plan9, inferno-os/utils (and plan9port), golang
%   (+elf linux + mach macos + pe windows + dwarf gdb),
%   kencc/9cc, +arm64, +riscv, 
%   9front
% - yet another fork, hmmm
%   but trying to incorporate all (golang Elf/Mach/PE, dwarf), latest
%   arm64/riscv, that can compile on arm64/amd64 (long vs int32),
%   that can cross-compile archs and OSes!
% - a self-supporting repo; embedded mk/rc/ed (better than inferno-os)
%   (and yacc?), Docker, GHA, good test suite with principia-softwarica
%   to check regressions
% - current state: 
% - future work: Mach-O, PE, multi-OS libc. Code is there
%   for Go, and actually part of goken9cc repo, but need to
%   remove go-specific and make real minilibc.

\section{Conclusion}

\end{document}
